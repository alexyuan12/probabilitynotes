\documentclass{article}
\usepackage{graphicx} % Required for inserting images
\usepackage{amsmath,amssymb,amsthm, graphicx}
\title{Probability and Measure Notes}
\author{Alexander Yuan}
\date{September 22 2025}

\begin{document}

\maketitle

\section{Measure Theory}
\subsection{Sigma Algebras}
Suppose $\Omega$ is a non-empty set and a collection $\mathcal{F} \subset \mathcal{P}(\Omega)$ satisfy: \newline \newline
 1) $\Omega \in \mathcal{F}$ \newline \newline
 2) $\{A_i\}_{i \in \mathbb{N}} \subset \mathcal{F} \implies \bigcup_{i \in\mathbb{N}}A_i \in \mathcal{F}$ \newline \newline
 3) $A \in \mathcal{F} \implies A^c\in \mathcal{F} $ \newline \newline
Such a collection  $\mathcal{F}$ is called a $\sigma$-algebra over $\Omega$. A collection that satisfies 1) and 3) but is only closed under finite unions is called an \emph{algebra}. Some authors also use field instead of algebra. \newline \newline
A sigma algebra is closed under countably infinite intersections and an algebra is closed under finite intersections. This follows immediately from \emph{DeMorgan's law(s)}.\newline \newline
The \emph{$\sigma$-algebra generated by a class} $\mathcal{C}\subset \mathcal{P}(\Omega)$ denoted $\sigma(\mathcal{C})$, is referred to as the smallest sigma algebra containing $\mathcal{C}$ and is defined by taking the intersection of all sigma algebras that contain $\mathcal{C}$.

\subsubsection{Toy $\sigma$-algebra:}
Let $\Omega = \{a,b,c,d\}$, and consider the classes \[
\mathcal{F}_1 = \{\Omega, \emptyset, \{a\}\}
\]
\[
\mathcal{F}_2 = \{\Omega, \emptyset, \{a\}, \{b,c,d\}\}
\]
$\mathcal{F}_2$ is a $\sigma$-algebra (and an algebra), but $\mathcal{F}_1$ is neither. 

\subsubsection{Finite/Cofinite $\sigma$-algebra}
Let $\Omega$ be a nonempty set, and let $|A|$ denote the number of elements of a set $A\subset \Omega$. Define the collection $\mathcal{F}_3=\{A \subset \Omega: \text{ either } |A| \text{ is finite or } |A^c| \text{ is finite.}\}$
$\mathcal{F}_3$ is a $\sigma$-algebra if and only if $|\Omega|<\infty$.

\subsubsection{Lemma: If $\mathcal{C}\subset\sigma(\mathcal{D)}$ then $\sigma(\mathcal{C)}\subset\sigma(\mathcal{D)}$}
\textit{Proof.} \newline \newline
$\sigma(\mathcal{D)}$ is a sigma algebra that contains $\mathcal{C}$, it therefore also contains $\sigma(\mathcal{C)}$.


\subsubsection{Trivial $\sigma$-algebra(s)}
For any set $\Omega$, the collection $\{\emptyset,\Omega\}$, as well as the power set $\mathcal{P}(\Omega)$ is a sigma algebra. We can think of these respectively as the "smallest" and "largest" sigma algebras we can define on $\Omega$.

\subsubsection{Semi-Algebra}
Another collection of subsets of interest is the \emph{semi-algebra}. Let $\Omega$ be a non-empty set. A collection $\mathcal{C} \subset \mathcal{P}(\Omega)$ is called a semi-algebra if: 
\[
1) \ A,B \in \mathcal{C} \implies A \cap B \in \mathcal{C}
\]
\[
2) \text{ For any } A \in \mathcal{C}, \text{ there exist sets } B_1,\ldots,B_k \in \mathcal{C} \text{ such that } A^c = \bigcup_{i=1}^k B_i
\]

\subsubsection{Half-open/ray Semi-algebra}
Let $\Omega = \mathbb{R}$ and $\mathcal{C}:=\{(a,b], (b, \infty): -\infty \leq a,b < \infty\}$, then $\mathcal{C}$ is a semi-algebra.

\subsubsection{Interval Semi-algebra}
Define an interval in $\mathbb{R}$ as a set $I\subset \mathbb{R}$ such that $a,b\in I, a<b \implies (a,b) \subset I.$ Let $\Omega = \mathbb{R}$ and $\mathcal{C}:= \{I: I \text{ is an interval} \}$, then $\mathcal{C}$ is a semi-algebra.
\subsubsection{Borel $\sigma$-algebra}
If $\Omega$ is a metric space (or more generally any topological space), then the $\sigma$-algebra generated by the topology of $\Omega$ is called the Borel $\sigma$-algebra on $\Omega$. It contains all open sets and closed subsets of $\Omega$, all countable intersections of open sets, and all countable unions of closed sets.\newline \newline
We denote the Borel $\sigma$-algebra on $\mathbb{R}$ as $\mathcal{B}_{\mathbb{R}}$. For example, if $\Omega$ is Hausdorff, then $\mathcal{B}_{\Omega}$ contains all countable and co-countable sets. 

\subsubsection{Borel $\sigma$-algebra generation}
The open intervals, closed intervals, half open intervals, open and closed rays all generate $\mathcal{B}_{\mathbb{R}}$. When we construct the Borel Measure, it will be the half-open intervals of most interest.\newline \newline
\textit{Proof.} \newline \newline
(Verify this!). 

\subsubsection{Product $\sigma$-algebra}
Let $\{X_\alpha\}_{\alpha\in A}$ be an indexed collection of non empty sets and $\Omega=\prod_{\alpha\in A}X_{\alpha}$ and coordinate maps $\pi_{\alpha}:\Omega\to X_{\alpha}$. If $\mathcal{F}_{\alpha}$ is a $\sigma$-algebra on $X_{\alpha}$ for each $\alpha$, the \emph{product $\sigma$-algebra} on $\Omega$ is the $\sigma$-algebra generated by:
\[
\{ \pi^{-1}(E_{\alpha}):E_{\alpha}\in \mathcal{F_{\alpha}}, \ \alpha  \in A\}
\]
We denote this by $\bigotimes_{\alpha\in A}\mathcal{F}_{\alpha}$. This definition is a little abstract, keep in mind it is incredibly significant for  maps between measurable spaces.

\subsubsection{Countably generated $\sigma$-algebra}
If the index set A for $\{X_\alpha\}_{\alpha\in A}$ is countable, then $\bigotimes_{\alpha\in A}\mathcal{F}_{\alpha}$ is the $\sigma$-algebra generated by $\{\prod_{\alpha \in A}E_{\alpha}:E_{\alpha} \in \mathcal{F}_{\alpha}\}$. \newline \newline
\textit{Proof.} \newline \newline
If $E_{\alpha} \in \mathcal{F}_{\alpha}$, then $\pi^{-1}_{\alpha}(E_{\alpha})=\prod_{\beta \in A}E_{\beta}$ where $E_{\beta} = X_{\beta} \ \text{for} \ \beta \neq \alpha$. Also note $\prod_{\alpha \in A}E_{\alpha}=\bigcap_{\alpha \in A}\pi^{-1}_{\alpha}$ and the result follows from (1.1.1).

\subsubsection{Product (Cylindrical) $\sigma$-algebra generation}
Suppose that $\mathcal{F}_{\alpha}$ is generated by a class $\mathcal{E}_{\alpha}$, $\alpha \in A$. Then $\bigotimes_{\alpha \in A}\mathcal{F}_{\alpha}$ is generated by $\mathcal{J}_1$ := $\{\pi^{-1}_{\alpha}(E_{\alpha}):E_{\alpha}\in \mathcal{E},\ \alpha \in A \}$. If the index set A is countable and $X_{\alpha}\in \mathcal{E}_{\alpha}$ for all $\alpha$, $\bigotimes_{\alpha \in A} \mathcal{F}_{\alpha}$ is generated by $\mathcal{J}_2$ := $\{\prod_{\alpha \in A}E_{\alpha}:E_{\alpha} \in \mathcal{E}_{\alpha} \}.$\newline \newline
\textit{Proof.}\newline \newline
(Verify This!)\newline \newline
Note: The second assertion follows from the previous proposition.

\subsubsection{Product $\sigma$-algebra reduction}
Let $\Omega_1,\ldots,\Omega_n$ be metric spaces and let $\Omega=\prod_{i=1}^n\Omega_i$, equipped with the metric on the cartesian product of finitely many metric spaces. Note that this product metric $d$ on $\Omega_1 \times \Omega_2$ is defined by $d((x_1,x_2), (y_1, y_2))=\max\{d(x_1,y_1), d_2(x_2, y_2)\}$ Then $\bigotimes_{i=1}^n\mathcal{B}_{\Omega_i}\subset\mathcal{B}_{\Omega}$. Recall a metric space $\Omega_i$ is separable provided there is a coutable dense subset in $\Omega_i$. If the $\Omega_i$'s are separable, then $\bigotimes_{i=1}^n\mathcal{B}_{\Omega_i}= \mathcal{B}_{\Omega}$.\newline \newline
\textit{Proof.}\newline \newline
By the previous proposition, $\bigotimes_{i=1}^n\mathcal{B}_{\Omega_i}$ is generated by the sets $\pi_i^{-1}(U_i)$, for $1\leq i \leq n$ where $U_i$ is open in $\Omega_i$. Since these sets are open in $\Omega$, it follows that  $\bigotimes_{i=1}^n\mathcal{B}_{\Omega_i}\subset\mathcal{B}_{\Omega}$ by (1.1.1). Suppose that $C_i$ is a countable dense set in $\Omega$, and let $\mathbb{B}_i$ be the collection of (open) balls in $\Omega_i$ with rational radius and center in $C_i$. Then every open set in $\Omega_i$ is a countable union of members of $\mathbb{B}_i$. Moreover, the set of points in $\Omega$ whose ith coordinate is in $C_i$ for all i is a countable dense subset of $\Omega$, and the balls of radius r in $\Omega$ turn out to be products of balls of radius r in the $\Omega_i$'s. It follows that $\mathcal{B}_{\Omega_i}$ is generated by $\mathbb{B}_i$ and $\mathcal{B}_{\Omega}$ is generated by $\{\prod_{i=1}^nB_i:B_i\in\mathbb{B}_i\}$. Therefore, $\bigotimes_{i=1}^n\mathcal{B}_{\Omega_i}=\mathcal{B}_{\Omega}$.

\subsubsection{Corollary (Borel $\sigma$-algebra on $\mathbb{R}^n$)}
$\mathcal{B}_{\mathbb{R}^n}=\bigotimes_{i=1}^n\mathcal{B}_{\mathbb{R}}$\newline \newline
\textit{Proof.}\newline \newline
Follows from the previous proposition.

\subsection{Measures}
Note that we will define a measure with a $\sigma$-algebra as its domain, however very often we may want to start with a measure defined on an algebra $\mathcal{A}$ and then extend it to a measure on $\sigma(\mathcal{A})$. We define a measure only defined on an algebra to be a \emph{pre-measure}. Similarly, one can begin with a definition of a measure on a class of subsets of $\Omega$ that only form a \emph{semi-algebra}.

\subsubsection{Measure Spaces:} Let $\Omega$ be a set equipped with a $\sigma$-algebra $\mathcal{F}$. A measure $\mu$ on a measurable space $(\Omega, \mathcal{F})$ is a function $\mu: \mathcal{F} \to [0, \infty]$ such that:
\[
1) \ \mu(\emptyset) = 0
\]
\[
2) \ \{E_n\}_{n \in \mathbb{N}}, \ E_j \cap E_k =\emptyset, \ j\neq k \implies \mu(\bigcup_{n\in \mathbb{N}}E_n) = \sum_{n \in \mathbb{N}}\mu(E_n)
\]
We call 2) countable additivity, and refer to  $(\Omega, \mathcal{F}, \mu)$ as a \emph{measure space}. \newline \newline
The following is some standard terminology concerning the ``size'' of $\mu$ on $(\Omega,\mathcal{F})$. If $\mu(\Omega)<\infty$, we say that $\mu$ is finite. If $\Omega = \bigcup_{i=1}^{\infty}A_i$ and $\mu(A_i)< \infty$ for all i, we say that $\mu$ is $\sigma$-finite.  \newline \newline

\subsubsection{Lemma (Disjointification):} From any sequence $\{A_n\}_{n \in \mathbb{N}} \subset \mathcal{P}(\Omega)$, one can construct a pairwise disjoint sequence $\{B_n\}_{n \in \mathbb{N}} \subset \mathcal{P}(\Omega)$ defined by $B_1=A_1$ and for $n \geq 2:$
\[
B_n = A_n \cap (\bigcup_{i=1}^{n-1}A_i)^c
\]
with the same union 
\[
\bigcup_{n \in \mathbb{N}}A_n = \dot{\bigcup}_{n\in \mathbb{N}}B_n
\]
where $\dot{\bigcup}$ denotes a disjoint union.\newline \newline
\textit{Proof.} \newline \newline
(Verify This!)

\subsubsection{Theorem (Properties of Measures):}
Let  $(\Omega, \mathcal{F}, \mu)$ be a measure space. We have: 
\[
1) \ A,B\in\mathcal{F}, \ A\subset B \implies \mu(A)\leq\mu(B)
\]
\[
2) \ A\subset \bigcup_{i=1}^{\infty}A_i \implies \mu(A)\leq\sum_{i=1}^{\infty}\mu(A_i)
\]
\[
3) \ A_i \uparrow A \implies \mu(A_i)\uparrow \mu(A)
\]
\[
4) \ A_i \downarrow A \implies \mu(A_i) \downarrow\mu(A)
\]
\textit{Proof:} \newline \newline
1) Note $B = A \cup (B\cap A^C)$ and use finite additivity: 
\[
\mu(B) = \mu(A)+\mu(B \cap A^c) \geq\mu(A)
\]
2) Set $A_j' = A_j \cap A$, $B_1=A_1'$ and $B_j=A'_j\cap(\bigcup_{j=1}^{n-1}A'_j)^c$ for $j>1$
\[
\mu(A) =  \sum_{i=1}^{\infty}\mu(B_i) \leq  \sum_{i=1}^{\infty}\mu(A_i)
\]
which follow from our disjointification lemma and 1). \newline \newline
3) Set $A_0 = \emptyset$, 
\[
\mu(\bigcup_{i=1}^{\infty}A_i) = \sum_{i=1}^{\infty} \mu(A_i \cap A_i^c) =\lim_{n\to \infty}\sum_{i=1}^{n}\mu(A_i \cap A_i^c) = \lim_{n\to \infty}\mu(A_n).
\]\newline \newline
4) Set $B_j= A_1\cap A_j^c$ and observe $\mu(A_1) = \mu(B_j)+\mu(A_j)$ then by 3):
\[
\mu(A_1) = \mu(\bigcap_{i=1}^{\infty}A_i) +\lim_{i\to \infty}\mu(B_i)= \mu(\bigcap_{i=1}^{\infty}A_i)+\lim_{i\to \infty}(\mu(A_1)-\mu(A_i))
\]
Subtract $\mu(A_1)$ and we get the desired result.

\subsubsection{Dirac Measure}
Let $(\Omega, \mathcal{F})$ be a measurable space. Let $w\in \mathcal{F}$, then the \emph{Dirac Measure} at w is 
\[
\delta_w(E):=\begin{cases}
0, \ w\notin E\\
1, \ w \in E
\end{cases}
\]

\subsubsection{Discrete Measure}
If $(\Omega, \mathcal{F}, \mu)$ is a measure space, then an \emph{atom}  is a subset $A \subset\Omega\in \mathcal{F}$ such that: 
\[
1) \ \mu(A) = A > 0.
\]
\[
2) \ \forall B \subset A, \mu(B)=A \ or \ \mu(B)=0
\]
The measure space is called \emph{discrete} if we can write: 
\[
\Omega = Z \ \dot{\cup} \ (\dot{\bigcup}_{n\in \mathbb{N}}A_n)
\]
where $\mu(Z) = 0$ and $\{A_n\}_{n \in \mathbb{N}}$ is a collection of atoms. \newline \newline
We say $\mu$ is \emph{discrete} if and only if it is a series of dirac measures.

\subsubsection{Counting Measure}
The \emph{counting measure} $\mu$ assigns to any set X the cardinality of that set X: 
\[
\mu(X) :=|X|
\]
The \emph{counting measure} of any infinite set is $\infty$.

\subsubsection{Lebesgue Measure (sketch and facts)}
The \textit{Lebesgue measure} on $\mathbb{R}^n$ is a model of ``length'', ``area'', ``volume'' for n = 1, 2, 3 respectively. Many texts tend to define the \textit{Lebesgue Measure} of an interval in $\mathbb{R}$ as it's length. More importantly, the \textit{Lebesgue measure} has the property of \emph{translation invariance} meaning for any set $U\subset \mathbb{R}^n$, and any $v\in\mathbb{R}^n$: 
\[
\mu(U)=\mu(U+v)
\]
The \emph{Lebesgue measure} of a ball is 
\[
\mathcal{L}^n(B_{x,r})=\mathcal{L}^n(B_{0,1})r^n=\frac{2\pi^{n/2}r^n}{n\Gamma(n/2)}=\frac{1}{n}\mathcal{L}^n(\mathbb{S}^{n-1})r^n
\]
where 
\[
\mathcal{L}^n(\mathbb{S}^{n-1})= \frac{2\pi^{n/2}}{\Gamma(\frac{n}{2})}
\]
is the area of $\mathbb{S}^{n-1}$ which is the sphere of radius r = 1 in $\mathbb{R}^n.$ The classical measure of the ball can be extended in a countably additive way to a sigma algebra containing all balls which makes integration theory relatively painless. \newline \newline
This discussion is continued more once we learn more about Borel Measures.

\subsection{Vitali Sets}
\subsubsection{Why not the Power Set?}
A natural question to ask is why not take the power set as our sigma-algebra each time? If our space $\Omega$ is at most countably infinite, then we have no issues in doing this.\newline \newline  However, if we are interested in uncountably infinite spaces like $\Omega = \mathbb{R}$ (or $\bar{\mathbb{R}}$), the prototypical counterexample is to try to define a measure that generalizes the notion of the length of an interval $I\subset\mathbb{R}$ on the entirety of $\mathcal{P}(\mathbb{R})$. \newline \newline
It is impossible to demand a set function $\lambda$ satisfy: 
\[
1) \ \lambda :\mathcal{P}(\mathbb{R}) \to [0, \infty]
\]
\[
2) \ \lambda((a,b]) = b-a
\]
\[
3) \ A \subset \mathbb{R} \implies \forall x \in \mathbb{R}:
\lambda(A) = \lambda(\{a+x:a\in A\})
\]
\[
4) \ \{A_i\}_{i \in \mathbb{N}} \subset 2^{\mathbb{R}},\ A_m \cap A_m = \emptyset, \ n\neq m \implies \lambda(\dot{\bigcup_{i=1}^{\infty}}A_i) = \sum_{i=1}^{\infty}\lambda(A_i)
\]
Conditions 2), 3), and 4) are non negotiable for a function that is supposed to capture length. (Verify this!). We now show the existence of a set which one cannot assign a notion of ``size''.

\subsubsection{The Vitali Set}
A \emph{coset of $\mathbb{Q}$} in $\mathbb{R}$ to be any set of the form:
\[
x+\mathbb{Q}=\{x+q\ | \ q \in \mathbb{Q}\}
\]
where $x\in \mathbb{R}$. The cosets of $\mathbb{Q}$ form a partition of $\mathbb{R}$ and also happen to be dense in $\mathbb{R}$. We denote the collection of all the cosets of \textit{$\mathbb{Q}$} in $\mathbb{R}$ as $\mathbb{R}/\mathbb{Q}$. It turns out that these cosets can be used to create structures on $\mathbb{R}$ that violate our geometric intuition. \newline \newline 
A subset $V\subset[0,1]$ is called a \emph{Vitali} set if V contains a single point from each coset of \textit{$\mathbb{Q}$} in $\mathbb{R}$. We can ``construct'' a \emph{Vitali set} using the \emph{axiom of choice}, simply by choosing one element of (x+$\mathbb{Q}$)\ $\cap \ [0,1]$ for each coset $x+\mathbb{Q} \in\mathbb{R}/ \mathbb{Q.}$ Of course, this “construction” is difficult to describe algorithmically, since we are making uncountably many arbitrary choices. \newline \newline 
The \emph{axiom of choice} says that every collection of non-empty, pair-wise disjoint sets has a \emph{choice set}. A \emph{choice set} for a set A is a set that contains exactly one member from each member of A.\newline \newline 
We will take it as a fact that V is not Lebesgue measurable. The main takeaway of the Vitali set should be that there does not exist such a measure satisfying all 4 properties we want a measure to have.
\subsection{Construction of Measures}

\subsubsection{Outer Measures}
The abstract generalization of the notion of an ``outer'' area is the \emph{outer measure} on a nonempty set $\Omega$: 
\[
\mu^*:\mathcal{P}(\Omega)\to[0,\infty]:
\]
that satisfies:
\[
\mu^*(\emptyset)
=0\]
\[
A \subset B \implies\mu^*(A)\leq\mu^*(B)\
\]
\[
\mu^*(\bigcup_{i=1}^{\infty}A_i)\leq \sum_{i=1}^{\infty}\mu^*(A_i)\
\]
The most common way to obtain outer measures is to start with a family $\mathcal{E}$ of ``elementary'' sets on which a measure is defined (such as rectangles in the plane) and then to approximate arbitrary sets ``from the outside'' by countable unions of members of $\mathcal{E}$. The next proposition gives a precise construction. 

\subsubsection{Outer Measure construction}
Let $\mathcal{E} \subset\mathcal{P}(\Omega)$ and f: $\mathcal{E} \to [0,\infty]$ such that $\emptyset, \ \Omega\in\mathcal{E}$ and $f(\emptyset)=0$. For any $A\subset \Omega$ define 
\[
\mu^*(A)=\inf \{\sum_{n=1}^{\infty}f(E_n): E_n \in \mathcal{E}, \ A\subset \bigcup_{n=1}^{\infty}E_n\}
\]
then $\mu^*$ is an outer measure.
\newline \newline
\textit{Proof.}\newline \newline
For any $A \in \Omega$ there exists a covering $\{E_i\}_{i=1}^{\infty}\subset \mathcal{E}$, $A\subset\bigcup_{i=1}^{\infty}E_i.$ We can take $E_i=\emptyset$ for all i showing $\mu^*(\emptyset) = 0$. Note that $\mu^*(A)\leq\mu^*(B)$ for $A\subset B$ because the set in which the infimum is taken over in the definition of $\mu^*(A)$ includes the corresponding set in $\mu^*(B)$,
Suppose $\{A_i\}_{i=1}^{\infty} \subset \mathcal{P}(\Omega)$ and let $\epsilon>0$. For each i, there exists $\{E_i^k\}_{k=1}^{\infty} \subset \mathcal{E}$ such that $A_i \subset \bigcup_{k=1}^{\infty}E^k_i$ and $\sum_{k=1}^{\infty}f(E^k_i) \leq \mu^*(A_i)+(2^{-i})\epsilon$ \newline \newline
Now if $\bigcup_{i=1}^{\infty}A_i=A$, we have $A\subset\bigcup_{i,k=1}^{\infty}E^k_i$ and $\sum_{i,k}f(E^k_i)\leq\sum_i\mu^*(A_i) +\epsilon$. Since $\epsilon$ is arbitrary we have shown countable subadditivity holds.

\subsubsection{Caratheodory-measurability}
If $\mu^*$ is an outer measure defined on $\mathcal{P}(\Omega)$, a set $A\subset \Omega$ is said to be \emph{$\mu^*$-measurable} (or \emph{Caratheodory measurable}) if:
\[
\mu^*(E) = \mu^*(E\cap A)+\mu^*(E\cap A^c) \ \text{for all} \ E \subset \Omega.
\]
Note one can define an \emph{``inner measure''} $\mu_*$ by $\mu_*(E)=\mu^*(\Omega)-\mu^*(E^c)$. If $\mu^*$ was induced from a countably additive measure defined on some algebra of sets in $\Omega$, then a subset of $\Omega$ will be \emph{Caratheodory measurable} if and only if its outer and inner measures agree.
\newline \newline 
The point to emphasize here is that the concept of ``inner area'' is redundant and can be defined in terms of the outer measure. Take note of this definition as it is fundamental in obtaining measures from outer measures.

\subsubsection{$\pi$ and $\lambda$ systems}
We start out with some preliminary definitions of different collections of sets. A collection $\mathcal{P}$ is said to be a $\pi$-system if: 
\[
A,B \in \mathcal{P} \implies A \cap B \in \mathcal{P}
\]
and a collection $\mathcal{L}$ is said to be a $\lambda$-system if: 
\[
1) \ \Omega\in \mathcal{L}
\]
\[
2) \ A,B \in \mathcal{L}, \ A\subset B \implies B \cap A^c \in \mathcal{L}
\]
\[
3) \ A_n\in \mathcal{L}, \ A_n \uparrow A \implies A \in \mathcal{L} \ 
\]
A class $\mathcal{C}$ of intervals in $\mathbb{R}$ is a $\pi$-system whereas the class of all open discs (balls) in $\mathbb{R}^2$ is not. By convention, we assume that these classes both contain $\emptyset$. To see this for $\mathbb{R}^2$ note that $\mathbb{B}((0,0),1)\cap\mathbb{B}((1,0),1)$ is not a ball. \newline \newline
An important thing to keep in mind is that every $\sigma$-algebra is a $\lambda$-system, but an algebra need not be a $\lambda$-system.

\subsubsection{Theorem ($\pi-\lambda$):}
If $\mathcal{P}$ is a $\pi$-system and $\mathcal{L}$ is a $\lambda$-system that contains $\mathcal{P}$, then $\sigma(\mathcal{P})\subset\mathcal{L}$. \newline \newline 
\textit{Proof:} \newline \newline
We first propose that if  $\mathcal{L}(\mathcal{P})$ is the $\lambda$-system generated by $\mathcal{P}$, then $\mathcal{L}(\mathcal{P})$ is a $\sigma$-algebra. To see this note, $\sigma(\mathcal{P})\subset \mathcal{L}(\mathcal{P})\subset \mathcal{L}$. It suffices to show that $\mathcal{L}(\mathcal{P})$ is a $\pi$-system.\newline \newline
First, define $\mathcal{G}_A :=\{B:A\cap B\in \mathcal{L}(\mathcal{P})\}$. \newline \newline
It follows that $\Omega \in \mathcal{G}_A$ since $A \in  \mathcal{L}(\mathcal{P})$. Further, if:
\[
B,C \in \mathcal{G}_A \ \text{such that} \ C \subset B
\] 
the
\[
A \cap (B\cap C^c) = (A\cap B) \cap (A\cap C)^c \in \mathcal{L}(\mathcal{P})
\]
since $A\cap B \ \text{and}\  A\cap C \in \mathcal{L}(\mathcal{P})$ and $\mathcal{L}(\mathcal{P})$ is a $\lambda$-system. \newline \newline
Lastly, 
\[
B_n \in \mathcal{G}_A \ \text{and} \ B_n \uparrow B
\]
then $A \cap B_n \uparrow A \cap B \in \mathcal{L}(\mathcal{P})$ since $A \cap B_n \in \mathcal{L}(\mathcal{P})$.
Since $\mathcal{P}$ is a $\pi$-system, 
\[
A \in \mathcal{P} \implies \mathcal{P} \subset\mathcal{G}_A \ \text{hence} \ \mathcal{L}(\mathcal{P}) \subset\mathcal{G}_A.
\]
Now,
\[
A \in \mathcal{L}(\mathcal{P}) \implies \mathcal{P} \subset\mathcal{G}_A \ \text{so we get that} \ \mathcal{G}_A \subset \mathcal{L}(\mathcal{P}) \  \text{as desired.}
\]
Note that $\pi-\lambda$ theorem has an equivalent statement called the \emph{monotone class theorem}.

\subsubsection{Complete Measures}
Let $(\Omega, \mathcal{F}, \mu)$ be a measure space. We say that $N\subset\mathcal{F}$ \text{is a \emph{null set} if} $\mu(N) = 0$. \newline \newline
If a statement $x\in \Omega$ is true except for some x in a null set we say that the statement is true \textit{almost everywhere}. Analogously, if our measure space is a probability space, we say the statement is true \emph{almost surely}. \newline \newline
We say $(\Omega, \mathcal{F}, \mu)$ is \emph{complete} if  $\mathcal{F}$ contains all subsets of sets of measure 0. Completeness can always be achieved by ``en-largening'' $\mathcal{F}$ as described by the following theorem below:
\subsubsection{Theorem (Completion of Measure Spaces):}
Suppose $(\Omega, \mathcal{M}, \mu)$ is a measure space. Let $\mathcal{N} := \{N \in\mathcal{M}: \mu(N) =0\}$ and $\bar{\mathcal{M}}:= \{E\cup F: E \in \mathcal{F}, \ F \subset N \ \text{for some}\  N\in \mathcal{N}\}$. Then $\bar{\mathcal{M}}$ is a $\sigma$-algebra, and there is a unique extension $\bar{\mu}$ of $\mu$ to a complete measure on $\bar{\mathcal{M}}$. \newline \newline
\textit{Proof.} \newline \newline
First note that since $\mathcal{M}$ and $\mathcal{N}$ are closed under countable unions, $\bar{\mathcal{M}}$ is too. If $E \cup F \in \bar{\mathcal{M}}$ where $E \in \mathcal{M}$ and $F \subset N$ for some $N \in \mathcal{N}$, we can then assume that $E\cap N = \emptyset$. Otherwise we can replace $F$ and $N$ by $F \cap E^c$ and $N \cap E^c$ respectively. We have that $E \cup F = (E \cup N )\cap (N^c \cup F)$, so $(E \cup F)^c = (E\cup N)^c \cup (N \cap F^c)$ But $(E \cup N)^c \in \mathcal{M}$ and $N \cap F^c\subset N$ so  $(E \cup F)^c \in \bar{\mathcal{M}}$ implying $\bar{\mathcal{M}}$ is a $\sigma$-algebra. \newline \newline
If $E \cup F \in \bar{\mathcal{M}}$ where $E \in \mathcal{M}$ and $F \subset N$ for some $N \in \mathcal{N}$, we set $\bar{\mu}(E\cup F) = \mu(E)$. We can check that this is well defined, since if $E_1 \cup F_1 = E_2 \cup F_2$, where $F_j\subset N_j\in \mathcal{N}$ then $E_1\subset E_2 \cup N_2$ and so $\mu(E_1) \leq \mu(E_2)+ \mu(N_2)=\mu(E_2)$, and likewise $\mu(E_2) \leq \mu(E_1)$. $\bar{\mu}$ is a complete measure on $\bar{\mathcal{M}}$ and is the only measure on $\bar{\mathcal{M}}$ that extends $\mu$.  (Verify This!) \newline \newline
We call $\bar{\mu}$ the \emph{completion} of $\mu$ and $\bar{\mathcal{M}}$ is called the \emph{completion} of $\mathcal{M}$ with respect to $\mu$.

\subsubsection{Theorem (Caratheodory Extension):}
If $\mu^*$ is an outer measure defined on $\mathcal{P}(\Omega)$ the collection of $\mu^*$-measurable sets is a $\sigma$-algebra and the restriction of $\mu^*$ to this collection is a complete measure.\newline \newline
Alternatively, if $\mu$ is a $\sigma$-finite measure defined on an algebra $\mathcal{A}$, then $\mu$ has a unique extension to $\sigma(\mathcal{A})$. \newline \newline
\textit{Proof.}\newline \newline
Let $\mathcal{P}$ be a $\pi$-system. If $\mu_1$ and $\mu_2$ are measures (defined on $\mathcal{F}_1$ and $\mathcal{F}_2$) that agree on $\mathcal{P}$ and there exists $\{A_n\}_{n\in \mathbb{N}} \in \mathcal{P}$ with $A_n \uparrow \Omega$ and $\mu_i(A_n) < \infty$, then $\mu_1$ and $\mu_2$ agree on $\sigma(\mathcal{P})$. To prove uniqueness proceed as follows:\newline \newline
Let $A \in \mathcal{P}$ such that $\mu_1(A) = \mu_2(A) < \infty$,
\[
\mathcal{L}
 = \{B \in \sigma(\mathcal{P}):\mu_1(A\cap B) = \mu_2(A\cap B) \}
\]
Since $A \in \mathcal{P}$, we have $\mu_1(A) = \mu_2(A)$ and $\Omega \in \mathcal{L}$. If $C\subset B\in \mathcal{L}$: 
\[
\mu_1(A\cap (B\cap C^c))= \mu_1(A \cap B) - \mu_1(A\cap C)
\] 
\[
=\mu_2(A\cap B)-\mu_2(A\cap C) = \mu_2(A\cap (B\cap C^c))
\]
If $B_n\in \mathcal{L}, \ B_n \uparrow B$:
\[
\mu_1(A\cap B ) = \lim_{n \to \infty}\mu_1(A \cap B_n) = \lim_{n \to \infty}\mu_2(A \cap B_n) = \mu_2(A\cap B ) 
\]
by continuity from below.\newline \newline
$\pi-\lambda$ implies $\sigma(\mathcal{P}) \subset\mathcal{L}$. Letting $A_n \in \mathcal{P},\ A_n \uparrow \Omega$, $\mu_1(A_n)=\mu_2(A_n)<\infty$, using the last result along with measure continuity we reach the uniqueness conclusion. \newline \newline
For existence, we must show that an arbitrary measure defined on an algebra $\mathcal{A}$ has an extension to $\sigma(\mathcal{A})$. Let $\mu^*$ be an outer measure. We now state two useful lemmas that are sufficient to show existence: \newline \newline
1) If $A \in \mathcal{A}$, then $\mu^*(A) = \mu(A)$ and A is $\mu^*$-measurable. \newline \newline
2) The class $\mathcal{A}^*$ of $\mu^*$-measurable sets is a $\sigma$-algebra and the restriction of $\mu^*$ to $\mathcal{A}^*$ is a measure. \newline \newline
To prove 1), without loss of generality assume $\mu^*(F) < \infty$ and let $\epsilon >0$, there exists a $\{B_i\}_{i \in \mathbb{N}} \in \mathcal{A}$ such that $F \subset \bigcup_{i=1}^{\infty}B_i$:
\[
\sum_{i=1}^{\infty}\mu(B_i) \leq \mu^*(F) + \epsilon
\]
Since $\mu$ is additive on $\mathcal{A}$, $\mu = \mu^*$ on $\mathcal{A}$ we have:
\[
\mu(B_i) = \mu^*(B_i\cap A) + \mu^*(B_i\cap A^c)
\]
Finally, summing over i and using the sub-additivity of $\mu^*$:
\[
\mu^*(F\cap A) + \mu^*(F^c\cap A) \leq
\]
\[
\sum_{i=1}^{\infty}\mu^*(B_i\cap A^c) + \sum_{i=1}^{\infty}\mu^*(B_i\cap A)
\]
\[
\leq \mu^*(F) + \epsilon.
\] which proves 1) since $\epsilon$ was arbitrary.  \newline \newline
We note that if E is $\mu^*$-measurable then $E^c$ is too. Now if $E_1$ and $E_2$ are $\mu^*$-measurable, then both their intersection and union are $\mu^*$-measurable. To see this let G be any subset of $\Omega$:
\[
\mu^*(G\cap(E_1\cup E_2)) + \mu^*(G\cap(E_1\cup E_2)^c)
\]
\[
\leq \mu(G\cap E_1) + \mu(G\cap E_1^c \cap E_2) + \mu(G\cap E_1^c \cap E_2^c)
\]
\[
=\mu^*(G\cap E_1)+ \mu^*(G \cap E^c) = \mu^*(G)
\]
This proves the conclusion for the union. Observe $E_1\cap E_2=(E_1^c\cup E_2^c)^c$ and use the $\mu^*$-measurability for complements to prove the conclusion for the intersection. \newline \newline
Let $G \subset \Omega$ and $E_1, \ldots, E_n$ be disjoint $\mu^*$-measurable sets. Then 
\[
\mu^*(G \cap \bigcup_{i=1}^{\infty}E_i) = \sum_{i=1}^{\infty}\mu^*(G \cap E_i)
\]
To see this, define $F_m:=\bigcup_{i\leq m} E_i$ where $E_n$ is $\mu^*$-measurable, $E_n \subset F_n$ and $F_{n-1}\cap E_n = \emptyset$. so 
\[
\mu^*(G\cap F_n) = \mu^*(G\cap F_n\cap E_n) + \mu^*(G\cap F_n\cap E_n^c)
\]
\[
=\mu^*(G\cap E_n) = \mu^*(G \cap F_{n-1})
\]
Induction then gives us the desired ``$\mu^*$-additivity result''. \newline \newline
Now, we must show that if $\{E_i\}_{i\in\mathbb{N}}$ is $\mu^*$-measurable, then $\bigcup_{i\in \mathbb{N}}E_i$ is $\mu^*$-measurable. To this end we appeal to the disjointifcation lemma, $E'_i = E_i \cap (\bigcap_{j<i}E_j^c)$. Our previous results imply $E'_i$ is $\mu^*$-measurable, and let $F_n = \bigcup_{i=1}^nE_i$ which is also $\mu^*$ measurable. By monotonicity and $\mu^*$-additivity, we have:
\[
\sum_{i=1}^n \mu^*(G \cap E_i)+\mu^*(G\cap E^c) = \mu^*(G\cap F_n) + \mu^*(G\cap E^c)
\]
\[
\leq \mu^*(G\cap F_n^c) +\mu^*(G\cap F_n) = \mu^*(G)
\]
Let $n\to \infty$ and subadditivity gives us:
\[
\mu^*(G\cap E^c) +\mu^*(G\cap E) \leq \sum_{i=1}^{\infty} \mu^*(G \cap E_i)+\mu^*(G\cap E^c) \leq \mu^*(G)
\]
Finally, we show countable additivity. Let $E_1, \ldots, E_n$ be disjoint and  $\mu^*$-measurable. Let $F_n = \bigcup_{i=1}^nE_i$. By monotonicity and $\mu^*$-additivity, 
\[
\sum_{i=1}^n\mu^*(E_i) =\mu^*(F_n) \leq \mu^*(E)
\]
Letting $ n \to \infty$ and subadditivity gives us our result! \newline \newline
\textbf{Remark:} \newline \newline
If $\mu^*(B) = 0$ then for any $E \subset \Omega$ we have 
\[
\mu^*(E) \leq \mu^*(E\cap B) + \mu^*(E\cap B^c) \leq \mu^*(E).
\]
Hence $\mu^*$ restricted to the collection of $\mu^*$-measurable sets is a complete measure.

\subsubsection{Pre-Measures}
One of the immediate applications of Caratheodory's Theorem is the extension of the domain of a measure from algebras to sigma-algebras. More precisely, let $\mathcal{A} \subset\mathcal{P}(\Omega)$ be an algebra. A function $\mu_0:\mathcal{A} \to [0, \infty]$ is called a \textit{pre-measure} if 
\[
1) \ \mu_0(\emptyset) = 0
\]
\[
2) \ \{A_i\}_{i \in \mathbb{N}}, \ A_n \cap A_m = \emptyset ; \\\bigcup_{i=1}^{\infty}A_i \in \mathcal{A} \implies \mu_0(\bigcup_{i=1}^{\infty}A_i)=\sum_{i=1}^{\infty}\mu_0(A_i)
\]
The important thing to note is that $\mu_0$ induces an outer measure on $\Omega$, namely: 
\[
\mu^*(E)=\inf \{\sum_{n=1}^{\infty}\mu_0(A_n): A_n \in \mathcal{A}, \ E\subset \bigcup_{n=1}^{\infty}A_n\}
\]

\subsubsection{Outer-Measure/Pre-measure relation}
If $\mu_0$ is a premeasure on $\mathcal{A}$ and $\mu^*$ is defined by above then the restriction of $\mu^*$ to $\mathcal{A}$ is $\mu_0$ i.e. $\mu^*|\mathcal{A}=\mu_0$. It also follows that every set in $\mathcal{A}$ is $\mu^*$-measurable. \newline \newline
\textit{Proof.}\newline \newline
(Verify this!)\newline \newline
Can also be found in Folland (pg. 31).
\subsubsection{Theorem (Premeasure-Extension):}
Let $\mathcal{A}\subset\mathcal{P}(\Omega)$ be an algebra, $\mu_0$ a premeasure on $\mathcal{A}$, and $\mathcal{F}=\sigma(\mathcal{A})$. There exists a measure $\mu$ defined on $\mathcal{F}$ whose restriction to $\mathcal{A}$ is $\mu_0$ namely $\mu= \mu^*|\mathcal{F}$ where $\mu^*$ is given by the previous proposition. If $\nu$ is another measure defined on $\mathcal{F}$ then $\nu(E)\leq \mu(E) \ \text{for all} \ E \in \mathcal{F}$ with equality when $\mu(E) < \infty$. If $\mu_0$ is $\sigma$-finite, then $\mu$ is the unique extension of $\mu_0$ to a measure defined on $\mathcal{F}$.\newline \newline
\textit{Proof.}\newline \newline
The first assertion follows from Caratheordory's Theorem and the previous pre-measure proposition. If $E \in \mathcal{F}$ and $E \subset \bigcup_{i=1}^{\infty}A_i$ where $A_i \in \mathcal{A}$ then $\nu(E) \leq \sum_{i=1}^{\infty}\nu(A_i)=\sum_{i=1}^{\infty}\mu_0(A_i)$ hence $\nu(E)\leq \mu(E)$. If we set $A=\bigcup_{i=1}^{\infty}A_i$, we have:
\[
\nu(A) = \lim_{n\to\infty}\nu(\bigcup_{i=1}^nA_i) = \lim_{n\to\infty}\mu(\bigcup_{i=1}^nA_i) = \mu(A)
\]
Let $\epsilon>0$. If $\mu(E) < \infty$ we are able to choose $A_i's$ such that $\mu(A) < \mu(E) + \epsilon$ implying  $\mu(A \cap E^c)< \epsilon$ and 
\[
\mu(E) \leq \mu(A) = \nu(A)=\nu(E) + \nu(A\cap E^C) \leq \nu(E) + \mu(A\cap E^C) \leq \nu(E) + \epsilon
\]
Since $\epsilon$ is arbitrary, $\mu(E)=\nu(E)$. Finally suppose $\Omega =\bigcup_{i=1}^{\infty}A_i$ with $\mu_0(A_i)<\infty$, with $A_i's$ assumed disjoint. Then for any $E \in \mathcal{F}$
\[
\mu(E)=\sum_{i=1}^{\infty}\mu(E\cap A_i) = \sum_{i=1}^{\infty}\nu(E\cap A_i)=\nu(E)
\]
showing $\nu=\mu$.\newline \newline
The proof of the theorem yields more than the statement. $\mu_0$ can be extended to a measure on the algebra $\mathcal{F}^*$ of all $\mu^*$-measurable sets. 
\subsubsection{Sigma-finite completion}
Let $(\Omega, \mathcal{F},\mu)$ be a measure space and let $\mu^*$ be the outer measure induced by $\mu$ according to the pre-measure proposition. Define $\mathcal{F^*}$ to be the sigma algebra of all $\mu^*$-measurable sets, and $\bar{\mu}= \mu^*|\mathcal{F^*}$ (restriction of $\mu^*$ to $\mathcal{F^*})$. It follows if $\mu$ is $\sigma$-finite then $\bar{\mu}$ turns out to be the completion of $\mu$. \newline \newline
\textit{Proof.} \newline \newline
(Verify This!)
\subsection{Borel Measures}
We now dissuss a large family of measures on $\mathbb{R}$ whose domain is the Borel $\sigma$-algebra $\mathcal{B}_{\mathbb{R}}$. We call such measures \emph{Borel measures} on $\mathbb{R}$. The general guiding idea is that the measure of an interval is it's length. The generalization to $\mathbb{R}^n$ is also relatively straightforward. \newline \newline
Let $(\mathbb{R},\mathcal{B}_{\mathbb{R}}, \mu)$ be a measure space such that $\mu$ is a finite Borel measure. Define the distribution function of $\mu$ as $F(x) = \mu((-\infty,x])$. F is increasing by monotonicity and right continuous by the continuity from below property since $(x, \infty] = \bigcap_{n=1}^{\infty}(-\infty,x_n]$ whenever $x_n\downarrow x$.\newline \newline
If $a<b$, $(-\infty,b] =(-\infty,a] \cup (a,b]$ we get: 
\[
\mu((a,b])= F(b) - F(a)
\]
Our goal is now to construct a measure $\mu$ starting from a distribution function F. \newline \newline
The building blocks for this theory are left-open, right closed intervals (i.e sets of the form $(a,b]$ or $(a,\infty)$ or $\emptyset$, where $-\infty\leq a<b<\infty$. We call these sets h-interals. It follows that if $\mathcal{A}$ is the collection of finite disjoint unions of h-intervals, $\mathcal{A}$ is an algebra, and $\sigma(\mathcal{A})= \mathcal{B}_{\mathbb{R}}$.
\subsubsection{``H-interval" Pre-Measure}
Let $F: \mathbb{R}\to \mathbb{R}$ be increasing and right continuous. If $(a_i,b_i]$ for $i=1,\ldots,n$ are disjoint h-intervals, then $\mu_0$ defines a pre-measure on $\mathcal{A}$ if: 
\[
\mu_0(\bigcup_{i=1}^{\infty}(a_i,b_i])= \sum_{i=1}^{\infty}(F(b_i)-F(a_i)), \ \mu(\emptyset) = 0.
\]
\textit{Proof.} \newline \newline
We first check that $\mu_0$ is well defined. If $\{(a_i,b_i]\}_{n=1}^{n}$ are disjoint$^*$ and $\bigcup_{i=1}^n(a_i,b_i]=(a,b]$ then (potentially after relabeling) we have a partition, $a=a_1<b_1 = a_2<b_2=\ldots<b_n=b$, such that $\sum_{i=1}^n(F(b_i)-F(a_i))= F(b) - F(a)$. More generally, if $\{I_i\}_{i=1}^{n}$ $\{J_j\}_{j=1}^{n}$ are finite sequences of disjoint h-intervals such that $\bigcup_{i=1}^nI_i = \bigcup_{j=1}^nJ_j$:
\[
\sum_{i}\mu_0(I_i) =\sum_{i,j}\mu_0(I_i \cap J_j)= \sum_{j}\mu_0(J_j)
\]
which shows $\mu_0$ is well defined and finitely additive. \newline \newline
It remains to show that if $\{I_i\}_{i=1}^{n}$ is a sequence of disjoint h-intervals with $\bigcup_{i=1}^nI_i = \mathcal{A}$, then countable additivity holds. Since $\bigcup_{i=1}^{\infty}I_i$ is a finite union of h-intervals, $\{I_i\}_{i=1}^{\infty}$ can be partitioned into finitely many subsequences such that the union of the intervals in each subsequence is a single \emph{h-interval}. Considering each subsequence separately and using the finite additivity of $\mu_0$, we can assume that $\bigcup_{i=1}^{\infty}I_i$ is a h-interval I = (a,b]. We have:
\[
\sum_{i=1}^n\mu_0(I_i)=\mu_0(\bigcup_{i=1}^{n}I_i) \leq \mu_0(\bigcup_{i=1}^{n}I_i) + \mu_0( I \cap (\bigcup_{i=1}^{n}I_i)^c)=\mu_0(I)
\]
Letting $n \to \infty$, it follows $\sum_{i=1}^{\infty}\mu_0(I)\leq\mu_0(I)$. \newline \newline
Suppose a and b are finite and let $\epsilon>0$. F is right continuous, so there exists a $\delta$ such that F(a+$\delta$) - F(a) $<$ $\epsilon$ and if $I_i=(a_i,b_i]$ for each i there exists a $\delta_i$ (dependent on i) such that F($b_i+\delta_i$) - F($b_i$) $< (2^{-i})\epsilon$. Note that [a+$\delta$, b] $\subset$ $\bigcup_{i=1}^{\infty}(a_i,b_i+\delta_i)$. Since [a+$\delta$, b] is compact, it follows that $\bigcup_{i=1}^{\infty}(a_i,b_i+\delta_i)$ reduces to a finite subcover by definition. By discarding $(a_i,b_i+\delta_i)$ that aren't in the finite subcover and relabeling the index i, we can assume that $(a_1,b_1),\ldots,(a_N,b_N)$ form a cover for $[a+\delta,b]$ with $b_i+\delta_i\in(a_{i+1,},b_{i+1}+\delta_{i+1})$ for i=$1,\ldots,N$.\newline \newline
Now,
\[
\mu_0(I) < F(b) - F(a+\delta) +\epsilon
\]
\[
\leq F(b_N+\delta_N)-F(a_1) +\epsilon
\]
\[
=F(b_N+\delta_N)-F(A_N) + \sum_{i=1}^{N-1}[F(a_{i+1})-F(a_{i})] +\epsilon
\]
\[
\leq F(b_N+\delta_N)-F(A_N) + \sum_{i=1}^{N-1}[F(b_i+\delta_i)-F(a_{i})] +\epsilon
\]
\[
<\sum_{i=1}^N[F(b_i)+(2^{-i})\epsilon-F(a_i)] +\epsilon
\]
\[
<\sum_{i=1}^{\infty}\mu_0(I_i)
+2\epsilon
\]
Thus, since $\epsilon$ is arbitrary if a and b are finite we are done.

\subsubsection{Theorem (Existence of Borel Measure on $\mathbb{R}$):}
If F: $\mathbb{R}\to\mathbb{R}$ is a increasing, right continuous function, there is a unique Borel measure $\mu_F$ on $\mathbb{R}$ such that $\mu_F((a,b])=F(b) -F(a)$ for all a, b. If G is another increasing, right continuous function, we have $\mu_f=\mu_g$ if and only if $F - G$ is constant. Conversely, if $\mu$ is a Borel measure on $\mathbb{R}$ that is finite on all bounded Borel sets and we define:  
\[
F(x) =\begin{cases}
\mu((0,x]),\  \text{if}\ x>0 \\
0, \ \text{if}\ x=0\\
-\mu((x,0]), \ \text{if}\ x<0\\
    
\end{cases}
\]
then $F$ is increasing and right continuous, and $\mu = \mu_F.$\newline \newline
This theorem can also be developed using half-open intervals of the form $[a,b)$ and left continuous functions F. We also note that if $\mu$ is a finite Borel measure on $\mathbb{R}$, then $\mu =\mu_F$ on $\mathcal{A}$ where $F(x) = \mu((-\infty,x])$ is called the \emph{cumulative distribution function (cdf)} of $\mu$. The \emph{cdf} differs from the F specified in the theorem above by the constant $\mu((-\infty,0])$.  \newline \newline
\textit{Proof.}\newline \newline
Each $F$ induces a pre-measure  on the algebra $\mathcal{A}$ (collection of finite disjoint h-intervals) by the previous proposition. It is clear that $F$ and $G$ induce the same pre-measure if and only if $F-G$ is constant, and $\mathbb{R}=\bigcup_{i=-\infty}^{\infty}(i,i+1]$ which implies that these pre-measures are $\sigma$-finite. The first two assertions follow from the \emph{pre-measure extension theorem}. The monotonicity of $\mu$ implies the monotonicity of $F$, similarly the continuity of $\mu$ from above and below imply the right continuity of F for $x\geq0$ and $x<0$. Finally, $\mu=\mu_F$ on $\mathcal{A}$ implies $\mu =\mu_F$ on $\mathcal{B}_{\mathbb{R}} $ by the uniqueness in the \emph{pre-measure extension theorem}.

\subsubsection{Lebesgue-Stijeles Measure}
The \emph{pre-measure extension theorem} implies that for each increasing and right continuous function F, we obtain not only the Borel measure $\mu_F$, but a complete measure $\bar{\mu}_F$ whose domain includes $\mathcal{B}_{\mathbb{R}}$. $\bar{\mu}_F$ is the completion of $\mu_F$, with a domain that is always strictly larger than $\mathcal{B}_{\mathbb{R}}$. Usually, this complete measure is also dentoed by $\mu_F$ and is called the \emph{Lebesgue-Stijeles} measure associated to F. \newline \newline
Fix a complete \emph{Lebesgue-Stijeles} measure $\mu$ on $\mathbb{R}$ associated to the increasing, right continuous function F, and we denote $\mathcal{F}_\mu$, the domain of $\mu$. It follows that for any $E\in \mathcal{F}_\mu$:
\[
\mu(E)=\inf\{\sum_{i=1}^{\infty}[F(b_i)-F(a_i)]:E\subset\bigcup_{i=1}^{\infty}(a_i,b_i]\}
\]
\[
=\inf\{\sum_{i=1}^{\infty}\mu((a_i,b_i]):E\subset\bigcup_{i=1}^{\infty}(a_i,b_i]\}
\]

\subsubsection{Lemma (Open Lebesgue-Stijeles Cover):}
For any $E  \in \mathcal{F}_\mu$:
\[
\mu(E) = \inf\{\sum_{i=1}^{\infty}\mu((a_i,b_i]):E\subset\bigcup_{i=1}^{\infty}(a_i,b_i]\}
\]
\[
=\inf\{\sum_{i=1}^{\infty}\mu((a_i,b_i)):E\subset\bigcup_{i=1}^{\infty}(a_i,b_i)\}
\]
\textit{Proof.}\newline \newline
(Verify This!) 

\subsubsection{Theorem (Lebesgue-Stijeles Regularity):}
The \emph{Lebesgue-Stijeles measure} satifies some nice regularity properties consistent with our intuition from calculus. In particular, a Lebesgue-Stijeles measurable set can be approximated from both inside and outside by compact sets or open sets.  \newline \newline
More precisely, if $E  \in \mathcal{F}_\mu$, then:
\[
\mu(E)=\inf\{\mu(U) : E\subset U, \ \text{U is open}\}
\]
\[
=\sup\{\mu(K) : K\subset E, \ \text{K is compact}\}
\]
\textit{Proof.} \newline \newline
By the previous lemma, for any $\epsilon$ there exist $(a_i,b_i)$'s such that $E\subset\bigcup_{i=1}^{\infty}(a_i,b_i)$ and $\sum_{i=1}^{\infty}\mu((a_i,b_i))\leq \mu(E)+\epsilon$. If $U=\bigcup_{i=1}^{\infty}(a_i,b_i)$ then $U$ is open, $E\subset U$, and $\mu(U)\leq\mu(E)+\epsilon$. Note $\mu(E) \leq\mu(U) $ whenever $E \subset U$, so we have shown the first equality. \newline \newline
Suppose first that $E$ is bounded. If E is closed, then E is compact by \emph{Heine-Borel} and the equality is obvious. Let $\epsilon>0$ and choose an open set U, $\bar{E}\cap E^c\subset U$ such that $\mu(U)\leq\mu(\bar{E}\cap E^c)+\epsilon$. Define $K=\bar{E}\cap U^c$. $K$ is compact, $K\subset E$ and it follows that 
\[
\mu(E)-\epsilon\leq\mu(E)-\mu(U)+\mu(\bar{E}\cap E^c)
\]
\[
\leq\mu(E)-(\mu(U)-\mu(U\cap E^c))=\mu(E) -\mu(E\cap U)
\]
\[
=\mu(K)
\]
If $E$ is unbounded, define $E_i=E\cap(i,i+1]$. For any $\epsilon>0$, there exists compact $K_i\subset E_i$ with $\mu(E_i)-(\frac{2^{-|i|}}{3})\epsilon\leq\mu(K_i)$. Define $H_n=\bigcup_{i=-n}^n{K_i}$. $H_n$ is compact, $H_n\subset E$ and $\mu(\bigcup_{i=-n}^n{E_i})-\epsilon\leq\mu(H_n)$. Note $\mu(E)=\lim_{n\to\infty}\mu(\bigcup_{i=-n}^n{E_i})$ and the result follows. 

\subsubsection{Theorem (Borel Sets and Measure Zero):}
If $E \subset\mathbb{R}, $ the following are equivalent:
\[
1) \ E \in \mathcal{F}_\mu
\]
\[
2) \ E = V\cap N_1 \text{ where } V \text{ is a countable intersection of open sets and } \mu(N_1)=0
\]
\[
3) \ H = V\cup N_2 \text{ where } H \text{ is a countable union of closed sets } \mu(N_2)=0
\]
\textit{Proof.}\newline \newline 
(Verify this!)
\newline \newline 
This theorem says that all Borel measurable sets (or more generally sets in $\mathcal{F}_\mu$) are of a reasonably simple form modulo sets of measure zero.

\subsubsection{Open interval approximation:}
If $E \in \mathcal{F}_\mu$ and $\mu(E)<\infty$, then for every $\epsilon>0$, there is a set A that is a finite union of disjoint open intervals such that
\[
\mu((A\cap B^c) \cup (A^c\cap B)) < \epsilon
\]
\textit{Proof.}\newline \newline 
(Verify this! )

\subsection{Lebesgue Measure}
The \emph{Lebesgue measure} denoted $m$ is the \emph{completion} of the \emph{Borel Measure}. The domain of m is called the class of \emph{Lebesgue measurable sets} denoted as $\mathcal{L}$. We also refer to the restriction of $m$ to $\mathcal{B}_{\mathbb{R}}$ as the \emph{Lebesgue measure}. Define:
\[
1) \ E+s:=\{x+s:x\in E\}
\]
\[
2) \ rE:=\{rx:x\in E\}
\]
where $E\subset \mathbb{R},\ and\ s,r\in\mathbb{R}, $ as \emph{translation invariance} and \emph{``simple'' behavior under dilations}.

\subsubsection{Theorem (Lebesgue Translation and Dilation):}
If $E\in \mathcal{L}$, then $E+s$ and $rE\in\mathcal{L}$ for all $s,r\in\mathbb{R}$. Moreover $m(E+s)=m(E)$ and $m(rE)=|r|m(E)$\newline \newline 
\textit{Proof.}\newline \newline 
Since the collection of open intervals is invariant under translations and dilations, $\mathcal{B}_{\mathbb{R}}$ is too. For $E\in\mathcal{B}_{\mathbb{R}}$, define $m_s(E)=m(E+s)$ and $m^r(E)=m(rE)$ then  $m_s(E)$ and $m^r(E)$ agree with $m$ and $|r|m$ on finite unions of intervals, hence on $\mathcal{B}_{\mathbb{R}}$ by our \emph{pre-measure extension} theorem. In particular, if $E\in\mathcal{B}_{\mathbb{R}}$ and $m(E)=0,$ then $m(E+s)=m(rE)=0$ which implies that the class of sets of \emph{Lebesgue measure} zero is preserved by translations and dilations. It also follows that the collection of \emph{Lebesgue measurable sets} (members of which are a union of a \emph{Borel set} and a \emph{Lebesgue} null set) is preserved by translations and dilations and that $m(E+s)=m(E)$ and $m(rE)=|r|m(E)$ for all \emph{Lebesgue} measurable sets $E$.

\subsubsection{Measure/Topological relation of subsets of $\mathbb{R}$}
Consider the following facts. Every singleton set in $\mathbb{R}$ has \emph{Lebesgue measure} zero, and hence so does every countable subset. In particular, $m(\mathbb{Q})=0$. The most surprising fact is that one can find a countable dense subset of $\mathbb{R}$ that has \emph{Lebesgue measure} 0. \newline \newline 
To make this precise, if we let $\{r_i\}_{i\in\mathbb{N}}$ be an enumeration of $\mathbb{Q}\cap[0,1]$, and given $\epsilon>0$, let $I_i$ be the open interval centered at $r_i$ of length $(2^{-j})\epsilon$. Then $U=(0,1)\cap\bigcup_{i=1}^{\infty}I_i$ is open and dense in $[0,1]$, but $m(U) \leq \sum_{i=1}^{\infty}(2^{-j})\epsilon=\epsilon$. The relative complement $[0,1]\cap U^c$ is closed and nowhere dense, but $1-\epsilon\leq m(U^c)$. A set that is open and dense, hence topologically ``large'' in a sense, can be measure-theoretically small. On the contrary, a set that is nowhere dense, hence  topologically ``small'' in a sense, can be measure-theoretically large. However, note that a nonempty open set cannot have \emph{Lebesgue measure} zero. \newline \newline 
The Lebesgue null sets include not only all countable sets but examples of sets having the cardinality of the continuum (cardinality of $\mathbb{R}$). The standard example is the Cantor set, which is also of interest for other reasons.

\subsubsection{Cantor Set}
To be continued. \newline \newline 

\section{Integration Theory}

\subsection{Measurable Functions}
On any measure space there is an obvious notion of an integral for functions that are, in a suitable sense, locally constant with the nice property that it can be extended to an integral for more general functions. Any set function $f:X\to Y$ induces a mapping $f^{-1}: \mathcal{P}(Y) \to \mathcal{P}(X)$ called the \emph{inverse image}. For any $E\subset Y$, we define :
\[
f^{-1}(E)=\{x\in X:f(x) \in E\}
\] which preserves unions, intersections, and complements. Thus, if $\mathcal{Y}$ is a $\sigma$-algebra on Y, $\{f^{-1}(E):E \in \mathcal{Y}\}$ is a $\sigma$-algebra on X. If $(X, \mathcal{X})\text{ and } (Y, \mathcal{Y})$ are measurable spaces, a mapping $f:X\to Y$ is called  \emph{$(\mathcal{X}, \mathcal{Y})$-measurable} if $f^{-1}(E) \in \mathcal{X}$ for all $E \in \mathcal{Y}$. Note that the composition of measurable functions is measurable. In particular $f:\mathbb{R}\to\mathbb{C}$ is \emph{Lebesgue measurable} if it is $(\mathcal{L},\mathcal{B}_\mathbb{C})$-measurable, and \emph{Borel measurable} if it is $(\mathcal{B}_\mathbb{R},\mathcal{B}_\mathbb{C})$-measurable . \newline \newline
\textbf{Warning:} if $f,g:\mathbb{R}\to\mathbb{R}$ are \emph{Lebesgue measurable}, it does not follow that $f\circ g$ is \emph{Lebesgue} measurable, even if $g$ is assumed continuous! More specifically if $E\in \mathcal{B}_\mathbb{R},$ we have $f^{-1}(E)\in \mathcal{L}$, but there is no guarantee that $g^{-1}(f^{-1}(E))\in \mathcal{L}$ unless $f^{-1}(E) \in \mathcal{B}_\mathbb{R}$.However if $f$ is \emph{Borel measurable}, then $f\circ g$ is \emph{Lebesgue  measurable} or \emph{Borel measurable} whenever $g$ is.

\subsubsection{Measurability Check}
If $\mathcal{Y}$ is generated by $\mathcal{E}$, then $f:X\to Y$ is \emph{$(\mathcal{X}, \mathcal{Y})$-measurable} if and only if $f^{-1}(E)\in \mathcal{X}$ for all $E \in\mathcal{E}$.\newline \newline 
\textit{Proof.}\newline \newline 
(Verify This!) \newline \newline
This is a very useful check. Pay attention to it as it comes up and up again.

\subsubsection{Measurability Equivalence}
If $(X, \mathcal{F})$ is a \emph{measurable space} and $f:X\to\mathbb{R}$, then following are equivalent:
\[
a) \text{ $f$ is } \text{$\mathcal{F}-$measurable}
\]
\[
b) \ f^{-1}((a,\infty)) \in \mathcal{F} \text{ for all a}\in \mathbb{R}
\]
\[
c) \  f^{-1}([a,\infty)) \in \mathcal{F} \text{ for all a}\in \mathbb{R}
\]
\[
d) \  f^{-1}((-\infty,a)) \in \mathcal{F} \text{ for all a}\in \mathbb{R}
\]
\[
e) \  f^{-1}((-\infty,a]) \in \mathcal{F} \text{ for all a}\in \mathbb{R}
\]
\textit{Proof.} \newline \newline
We first state a preliminary result about the \emph{Borel} $\sigma$-algebra on $\mathbb{R}$.
$\mathcal{B}_\mathbb{R}$ is generated by each of the following:
\[
\mathcal{E}_1=\{(a,b):a<b\}
\]
\[
\mathcal{E}_2=\{[a,b]:a<b\}
\]
\[
\mathcal{E}_3=\{(a,b]:a<b\} \text{ or } \mathcal{E}_4=\{[a,b):a<b\}
\]
\[
\mathcal{E}_5=\{(a,\infty):a\in\mathbb{R}\} \text{ or } \mathcal{E}_6=\{(-\infty,a):a\in\mathbb{R}\}
\]
\[
\mathcal{E}_7=\{[a,,\infty):a\in\mathbb{R}\} \text{ or } \mathcal{E}_8=\{[-\infty,a):a\in\mathbb{R}\}
\]
All of these sets are \emph{Borel sets}, so by $(1.1.1)$ we get $\sigma(\mathcal{E}_j)\subset\mathcal{B}_\mathbb{R}$. $\mathcal{E}_3, \  \mathcal{E}_4 $ are both countable intersections of open sets. $\mathcal{E}_3$ for example, note $(a.b]=\bigcap_{n=1}^{\infty}(a+n^{-1},b+n^{-1})$. On the other hand, recall every open set in $\mathbb{R}$ is a countable union of open intervals, so again by  $(1.1.1)$, we get $\mathcal{B}_\mathbb{R}\subset\sigma(\mathcal{E}_1)$. $\mathcal{B}_\mathbb{R}\subset\sigma(\mathcal{E}_j)$ for $j\geq2$ can now be established by showing all open intervals are contained in $\sigma(\mathcal{E}_j)$ and applying (1.1.1). The rest of the cases are straightforward (Verify This!). \newline \newline
This result along with our \emph{measurability check} give us the desired result.
\subsubsection{Measurability on subsets}
If $(X,\mathcal{F})$ is a \emph{measurable} space, $E\in\mathcal{F}$, we say $f$ is measurable on E if $f^{-1}(B)\cap E \in\mathcal{F}$ for all \emph{Borel sets} B. Equivalently, $f|E \text{ is }\mathcal{F}_E$-measurable where $\mathcal{F}_E=\{G\cap E:G\in\mathcal{F}\}$. Given a set $X$, if $\{(Y_\alpha,\mathcal{N}_\alpha)\}_{\alpha\in A}$ is a family of measurable spaces, and $f_\alpha:X\to Y_\alpha$ for each $\alpha \in A$, there is a unique smallest $\sigma$-algebra on $X$ with respect to which the $f_\alpha$'s are all measurable, namely, the $\sigma$-algebra generated by the sets $f_\alpha^{-1}(E_\alpha)$ with $E_\alpha \in \mathcal{N}_\alpha, \ \alpha \in A$. It is called the $\sigma$-algebra generated by $\{f_\alpha\}_{\alpha \in A}$. In particular, if $X = \prod_{\alpha\in A}Y_\alpha,$ it is the \emph{product/cylindrical} $\sigma$-algebra generated by the coordinate maps $\pi_\alpha:X\to Y_\alpha$

\subsubsection{Measurability of maps}
Let $(X,\mathcal{F}), \ (\prod_{\alpha\in A}Y_\alpha, \bigotimes\mathcal{N}_\alpha)$ be a measurable space, and $\pi_\alpha:Y\to Y_\alpha$ be coordinate maps. Then $f:X\to Y$ is $(\mathcal{F},\mathcal{N})$-measurable if and only if $f_\alpha=\pi_\alpha \circ f$ is $(\mathcal{F},\mathcal{N}_\alpha)$-measurable for all $\alpha$. \newline \newline
\textit{Proof.} \newline \newline
If $f$ is measurable, so is each $f_\alpha$ since the composition of measurable maps is measurable. Conversely, if each $f_\alpha$ is measurable, then for all $E_\alpha\in \mathcal{N}_\alpha, $$ f^{-1}(\pi_\alpha^{-1}(E_\alpha)) $ $= f^{-1}(E_\alpha)\in \mathcal{F}$. Thus, f is measurable by the \emph{measurability check}.
\newline \newline


\subsubsection{Corollary (Complex-Valued functions)}
A function $f:X\to \mathbb{C}$ is $\mathcal{F}$-measurable if and only if \textbf{Re}[$f$] and \textbf{Im}[$f$] are $\mathcal{F}$-measurable.\newline \newline
\textit{Proof.}\newline \newline
Note $\mathbb{C}$ is separable and homeomorphic to $\mathbb{R}\times\mathbb{R}$, hence $\mathcal{B}_\mathbb{C}=\mathcal{B}_{\mathbb{R}^2}=\mathcal{B}_\mathbb{R}\otimes\mathcal{B}_\mathbb{R}$ by the \emph{product $\sigma$-algebra reduction} proposition.

\subsubsection{Extended Real numbers}
The \emph{extended real numbers} is defined to be $\bar{\mathbb{R}}=\mathbb{R}\cup\{-\infty,+\infty\}$. The \emph{Borel sets} on $\bar{\mathbb{R}}$ are $\mathcal{B}_{\bar{\mathbb{R}}}= \{E\subset\ \bar{\mathbb{R}}:E\cap\mathbb{R}\}$. This coincides with the usual definition of $\mathcal{B}_\mathbb{R}$ if we consider $\bar{\mathbb{R}}$ as a metric space with $d(x,y)=|\arctan(x)-\arctan(y)|$. 

\subsubsection{Measurability of continuous functions}
If X and Y are topological spaces, every continuous function $f:X\to Y$ is ($\mathcal{B}_{\mathbb{X}}$,$\mathcal{B}_{\mathbb{Y}}$)-measurable.\newline \newline
\textit{Proof.}\newline \newline
This is a corollary to the \emph{measurability check}. Recall that $f$ is continuous if and only if $f^{-1}(U)$ is open in X for every open $U \subset Y$.

\subsubsection{``Algebra" of measurable functions}
If $f,g:X\to\mathbb{C}$ are both $\mathcal{F}$-measurable, then $f+g$ and $fg$ are $\mathcal{F}$-measurable. \newline \newline
\textit{Proof.}\newline \newline
Define $F:X\to \mathbb{C}\times\mathbb{C}, \ \phi:\mathbb{C}\times\mathbb{C}\to\mathbb{C}$ and $\psi:\mathbb{C}\times\mathbb{C}\to\mathbb{C}$ by:
\[
F(x)=(f(x),g(x)), \ \phi(z,w) = z+w, \ \psi(z,w) = zw
\]
Note $\mathcal{B}_{\mathbb{C}\times\mathbb{C}}=\mathcal{B}_{\mathbb{C}}\otimes\mathcal{B}_{\mathbb{C}}$ by the \emph{measurability check}, and that $F$ is $(\mathcal{F},\mathcal{B}_{\mathbb{C}\times\mathbb{C}})$-measurable by the previous proposition. It follows that $f+g=\phi\circ F$ and $fg=\psi\circ F$ are $\mathcal{F}$-measurable.

\subsubsection{``Limits" of measurable functions}
If $\{f_i\}$ is a sequence of $\bar{\mathbb{R}}$-valued functions on $(X,\mathcal{F})$ then the functions 
\[
g_1(x) = \sup_if_i(x), \ g_3(x) = \limsup_{i \to \infty}f_i(x)
\]
\[
g_2(x)= \inf_if_i(x), \ g_4(x) = \liminf_{i\to\infty}f_i(x)
\]
are all measurable. Furthermore, if $f(x) = \lim_{i\to \infty}f_i(x)$ exists for every $x\in X$, then $f$ is measurable. \newline \newline 
\textit{Proof.} \newline \newline
We have 
\[
g_1^{-1}((a, \infty])= \bigcup_{i=1}^{\infty}f_i^{-1}((a,\infty]), \ g_2^{-1}([-\infty,a))
\]
so $g_1$ and $g_2$ are both measurable by the \emph{measurability equivalence}\newline \newline 
More generally  if $h_k(x)= \sup_{i>k }f_i(x)$, then $h_k$ is measurable for each k, so $g_3=\inf_kh_k$ is measurable and likewise for $g_4$. Finally, if $f$ exists then $f=g_3=g_4$, so $f$ is measurable. \newline \newline
From these propositions, we get that if $f,g:X\to \mathbb{R}$ are \emph{measurable}, then so are the functions $\max(f,g) \text{ and } \min(f,g)$, and the fact that if $\{f_i\}_{i\in\mathbb{N}}$ is a sequence of complex-valued measurable functions, and $f(x)=\lim_{i\to \infty}f_i(x)$ exists for all $x$ then $f$ is measurable.

\subsubsection{Decompositions of functions}
For a function $f:X \to \bar{\mathbb{R}}$, define the \emph{positive} and \emph{negative} parts of $f$ to be 
\[
f^+(x)=\max (f(x),0), \ f^-(x)=\max(-f(x),0)
\]
Then $f=f^+-f^-$. If $f$ is measurable, then both the  \emph{positive} and \emph{negative} parts of $f$ are measurable by the discussion above. Second, if $f:X\to \mathbb{C}$, we have its \emph{polar} decomposition:
\[
f= sgn(f)|f|, \text{ where }sgn(f) = \begin{cases}\frac{f}{|f|} \text{ if } f \neq 0 \\0 \text{ if } f = 0
    
\end{cases}
\]

\subsubsection{Simple Functions}
Suppose $(X,\mathcal{F})$ is a measurable space, if $E\subset X$, the \emph{indicator function} $\mathbb{I}_E$ (which is sometimes called the \emph{characteristic function}) is defined by 
\[
\mathbb{I}_E=\begin{cases}
    1 \text{ if } x \in E \\ 0 \text{ if } x \notin E
    \end{cases}
\]
It is almost immediate that $\mathbb{I}_E$ is measurable if $E\in\mathcal{F}$. A \emph{simple function} on $X$ is a finite linear combination, with real (complex) coefficients, of \emph{indicator} functions of sets in $\mathcal{F}$. We do not allow \emph{simple functions} to assume the values $-\infty$ and $+\infty$. Equivalently, $f:X\to\mathbb{C}$ is \emph{simple} if and only if $f$ is measurable and the range of $f$ is a finite subset of $\mathbb{C}$. We call \[
f= \sum_{i=1}^nz_i\mathbb{I}_{E_i}, \text{ where } E_i= f^{-1}(\{z_i\}), \text{ and } \text{range}(f)=\{z_1,\ldots,z_n\}
\] 
the \emph{standard representation} of $f$. It exhibits $f$ as a linear combination with distinct coefficients, of characteristic functions of disjoint sets whose union is $X$. Note that if $f,g$ are \emph{simple}, then both $f+g$ and $fg$ are \emph{simple}. \newline \newline 
The following theorem is an important approximation result that shows arbitrary measurable functions can be ``nicely'' approximated by \emph{simple functions}.

\subsubsection{Theorem (Approximation by simple functions):}
Let $(X, \mathcal{F})$ be a measurable space. \newline \newline
1) If  $f:X \to [0,+\infty]$ is measurable, there is a sequence $ \{s_n\}$ of \emph{simple functions} such that $0 \leq s_1 \leq s_2 \leq\ldots \leq f, \ s_n\to f$ \text{ point-wise, and } $s_n\to f$ uniformly on any set on which is bounded.\newline \newline
2) If  $f:X \to \mathbb{C}$ is measurable, there is a sequence $ \{s_n\}$ of \emph{simple functions} such that $0 \leq |s_1| \leq |s_2|\leq\ldots \leq |f|, \ s_n\to f$ \text{ point-wise, and } $s_n\to f$ uniformly on any set on which is bounded.\newline \newline
\textit{Proof.}\newline \newline
To prove 1), note for every $n \in \mathbb{N}$, if $f$ is unbounded above then we may cut off the height at $n$, and then consider partitioning $[0,n]$ into intervals of equal height $2^{-n}$ and then approximating $f$ by its floor over this partition of the range. The following figure gives a pictoral demonstration for some of $s_n$ for some $n \in \mathbb{N}$(credit: Kebeseque on Discord). 
\begin{center}
    \includegraphics[width=1\linewidth]{Screenshot 2025-08-14 at 6.38.17 PM.png}
    \label{fig:placeholder}
\end{center} 
In other words for $n=0,1,2,\ldots,$ and $0\leq k\leq2^{2n}-1$, let:
\[
E^k_n=f^{-1}((k2^{-n,},(k+1)2^{-n,}]) \text{ and } F_n=f^{-1}((2^{-n},+\infty])
\] and define:
\[
s_n=\sum_{k=0}^{2^{n}-1}k2^{-n}(\mathbb{I}_{E_n^k})+2^n(\mathbb{I}_{F_n})
\]
Note: $s_n\leq s_{n+1}$ for all $n$ and $0 \leq f-s_n\leq2^{-n}$ on the set where $f \leq 2^n$.
\newline \newline
To prove 2), note if $f=g+ih$, we can apply part 1) to the \emph{positive} and \emph{negative} parts of $g$ and $h$, obtaining sequences $\psi^+_n, \ \psi^-_n, \ \zeta^+_n, \zeta^-_n$ of nonnegative simple functions that increase towards $g^+, \ g^-, \ h^+, \ h^-$. Let $\phi_n= \psi^+_n-\ \psi^-_n+ i( \zeta^+_n- \zeta^-_n)$ and note:
\[
\psi^+_n-\ \psi^-_n+ i( \zeta^+_n- \zeta^-)\to g^+-g^-+i(h^+-h^-)=g+ih=f
\]
(This will be fleshed out later, also need to finish showing $s_n\leq s_{n+1}$ for the last part of a).)

\subsubsection{Complete measurability}
Suppose $\mu$ is a complete measure, if $f$ is measurable and $f=g  \ \mu$-almost everywhere, then $g$ is measurable. Furthermore, if $f_n$ is measurable for $n \in \mathbb{N}$ and $f_n\to f$ $\mu$-almost everywhere, then $f$ is measurable. \newline \newline
\textit{Proof.}\newline \newline
Suppose  $\mu$ is a complete measure, $f$ is measurable and $f=g  \ \mu$-almost everywhere. Define $E:=\{x:f(x)\neq g(x)\}$. Suppose $A$ is measurable and note $g^{-1}(A) = (g^{-1}(A)\cap E) \cup (g^{-1}(A)\cap E^c)$. $(g^{-1}(A)\cap E)$ is measurable as it is contained in E which has a measure of 0. Finally, note that $(g^{-1}(A)\cap E^c)=(f^{-1}(A)\cap E^c)$ by assumption hence $g^{-1}(A)$ is measurable. \newline \newline
The second part is a similar idea (Verify this!)

\subsection{Integration of Non-Negative Functions}
Fix a measure space $(\Omega, \mathcal{F}, \mu)$, and let $L^+$ denote the space of all measurable functions from $\Omega$ to  $[0,\infty]$ If $\phi$ is a simple function in $L^+$ with standard representation $\phi = \sum_{i=1}^na_i\mathbb{I}_{E_i}$ we define the \emph{integral} of $\phi$ with respect to $\mu$ by 
\[
\int\phi d\mu= \sum_{i=1}^na_i\mu(E_i)
\]

\subsubsection{Integral properties}
Let $\phi, \ \psi$ be simple functions in $L^+$, then the following hold:
\[
\text{1) If } c\geq 0, \text{ then } \int c\phi = c\int \phi
\]
\[
\text{2) }\int (\phi+ \psi) = \int \phi+ \int\psi
\]
\[
\text{3) If } \phi \leq \psi, \text{ then } \int \phi \leq \int \psi
\]
\[
\text{ 4) the map } A\mapsto\int_A\phi d\mu \text{ is a measure on } \mathcal{F}.
\]
\textit{Proof.} \newline \newline
1) follows from the linearity of a sum. For 2), define $\sum_{j=1}^na_j\mathbb{I}_{E_j}$ and $\sum_{i=1}^nb_i\mathbb{I}_{F_i}$ to be the standard representations of $\phi$ and $\psi$. 
Then $E_j = \bigcup_{i=1}^m(E_j\cap F_i)$ and $F_i = \bigcup_{j=1}^m(E_j\cap F_i)$. Note that $\bigcup_{j=1}E_j=\bigcup_{i=1}F_i=\Omega$ where the unions are disjoint. Hence by finite additivity, 
\[
\int \phi + \int \psi = \sum_{j,k}(a_j+b_k)\mu(E_j \cap F_k)
\]
\[
=\int \psi + \phi
\]
Moreover, if $\psi \leq \phi$, then $a_j\leq b_k$ whenever $E_j \cap F_k \neq \emptyset $, so 
\[
\int \phi= \sum_{j,k}a_j\mu(E_j\cap F_k) \leq \sum_{j,k}b_k\mu(E_j \cap F_k) = \int \psi
\]
showing 3). Finally, let $\{A_k\}$ be a disjoint sequence in $\mathcal{F}$ and $A= \bigcup_{k=1}^\infty A_k$:
\[
\int_A\phi= \sum_ja_j\mu(A\cap E_j)= \sum_{j,k}a_j\mu(A_k \cap E_j)= \sum_k\int_{A_k}\phi
\]
which establishes 4).\newline \newline
From this, we can extend the integral to all functions $f \in L^+$ by defining 
\[
\int f d\mu = \sup \{ \int \phi d\mu: 0 \leq \phi \leq f, \phi \text{ simple}\}
\]

\subsubsection{Monotone Convergence Theorem}


\section{Probability Theory}
\subsection{Random Variables}
Let $(\Omega,\mathcal{F}, \mathbb{P})$ be a probability space. A function $X:\Omega\to\mathbb{R}$ is a random variable, if the event $X^{-1}((-\infty,a]):=\{\omega:X(\omega)\leq a\}\in\mathcal{F}$ for each $a \in \mathbb{R}$. This is equivalent to the stronger condition: $f^{-1}(A)\in\mathcal{F}$ for all \emph{Borel sets} $A\in\mathcal{B}_\mathbb{R}$  \newline \newline
Some argue that probability theory begins and measure theory ends with the definition of independence. If one only knows undergraduate probability theory, one's intuition might come from the easily envisioned property that the occurrence or non-occurrence of an event has no effect on our estimate of the probability that an independent event will or not occur.
Despite this intuitive appeal, it is important to recognize that independence is a technical concept with a technical definition which must be checked with respect to a specific probability model. There are many counterexamples to our naive ``intuition''. We now try to characterize independence through a few definitions.
\subsection{Independence}
Suppose $(\Omega, \mathcal{F},\mathbb{P})$ is a fixed probability space. We say the events $A, B \in \mathcal{F}$ are \emph{independent} if: \[
\mathbb{P}(A \cap B) = \mathbb{P}(A)\mathbb{P}(B)
\]
Two real random variables X and Y are \emph{independent} if for all $C,D\in\mathcal{B}_{\mathbb{R}}:$
\[
\mathbb{P}(X\in C,Y\in D)=\mathbb{P}(X\in C)\mathbb{P}(Y\in D)
\]
or in other words, the two events $A=\{A\in C\}$ $B=\{Y\in D\}$ are \emph{independent}.\newline \newline
Two $\sigma$-algebra's $\mathcal{F}$ and $\mathcal{G}$ are \emph{independent} if for all $A \in \mathcal{F}$ and $B \in \mathcal{G}$ the events A and B are \emph{independent}.\newline \newline
Let $\mathcal{C}_i\subset \mathcal{B}_{\mathbb{R}}$ for $i=1,\ldots,n$. The classes $\mathcal{C}_i$ are \emph{independent}, if for any choice $A_1,\ldots,A_n$, with $A_i\in\mathcal{C}_i$, we have that the events $A_1,\ldots,A_n$ are \emph{independent}.
\subsubsection{Theorem (Independence Criterion):}
If $\mathcal{C}_i$ is a non empty class of events for each $i=1,\ldots,n$, such that $\mathcal{C}_i$ is a $\pi$-system, and if $\mathcal{C}_i$'s are \emph{independent} for $i=1,\ldots,n$ 

\subsection{Radon-Nikodym Theorem}
Let $(\Omega,\mathcal{F},\nu)$ be a measure space and f be a non-negative Borel measurable function. The set function:
\[
\lambda(A)=\int_{A}fd\nu, \ A\in \mathcal{F}
\]
is a measure on $(\Omega,\mathcal{F})$. Note that $\nu(A)= 0 \text{ implies }\lambda(A)=0$. If $\nu(A)= 0 \text{ implies }\lambda(A)=0$ holds for $\lambda$ and $\nu$ defined on the same measurable space we say $\lambda$ is \emph{absolutely continuous} with respect to $\nu$. \newline \newline
Let $\nu$, $\lambda$ be two measures on  $(\Omega,\mathcal{F})$ and $\nu$ be $\sigma$-finite. If $\lambda$ is absolutely continuous with respect to $\nu$, then there exists a non-negative Borel measurable function $f$ on $\Omega$ such that:
\[
\lambda(A)=\int_{A}fd\nu, \ A\in \mathcal{F}
\]
is a measure on $(\Omega,\mathcal{F})$. Furthermore, $f$ is unique \emph{almost everywhere} $\nu$ or in other words $\lambda(A)=\int_{A}gd\nu$ for any $A\in \mathcal{F}$, then f=g \emph{almost everywhere} $\nu$. We call $f$ the \emph{Radon-Nikodym} derivative (or \emph{density}) of $\lambda$ with respect to $\nu$ denoted as $\frac{d\lambda}{d\nu}$. \newline \newline
\textit{Proof.}\newline \newline
Check Billingsley (might add conditional expectation proof)\newline \newline
A useful consequence of Radon-Nikodym is that if $f$ is Borel measurable on $(\Omega,\mathcal{F})$ and $\int_Afd\nu=0$ for any $A\in \mathcal{F}$, then $f=0$ almost everywhere. If $\int fd\nu=1 \ \text{for a} \ f \geq 0$ almost everywhere $\nu$, then $\lambda$ is given by above is a probability measure and we refer to $f$ as it's \emph{probability distribution function (pdf)}.\newline \newline
A continuous cumulative distribution function may not have a probability distribution function with respect to the Lebesgue measure. A necessary and sufficient condition is that $F$ is \emph{absolutely continuous} in the sense that for any $\epsilon >0$, there exists a $\delta>0$ such that for any finite collection of disjoint open intervals $(a_i,b_i)$, $\sum_i(b_i-a_i)<\delta$ implies $\sum[F(b_i)-F(a_i)] <\epsilon$. \emph{Absolutely continuity} in this sense is weaker than differentiability, but stronger than continuity.\newline \newline
 Any discontinuous cumulative distribution function is not \emph{absolutely continuous}, however every cumulative distribution function is differentiable almost everywhere Lebesgue measure. Thus, the important take away is that if $f$ is the probability distribution function of $F$ with respect to the Lebesgue measure, then $f$ is the usual derivative of $F$ almost everywhere Lebesgue measure and 
\[
F(x)=\int_{-\infty}^xf(y)dy, \ x  \in \mathbb{R}
\]
holds.
\subsection{Conditional Expectation}
In elementary probability, we are able to define the conditional probability of an event A given event B provided the probability of event B occurring is strictly greater than 0. However, we sometimes need a notion of ``conditional'' probability for events with probability 0. For example, define the event $B:=\{Y=c\}$ where $c\in \mathbb{R}$ and Y is a random variable with a continuous cdf.\newline \newline
Let X be an integrable random variable on $(\Omega,\mathcal{F},\mathbb{P})$. Let $\mathcal{D}$ be a sub-$\sigma$-algebra of $\mathcal{F}$. The \emph{conditional expectation} of X given $\mathcal{D}$ denoted $\mathbb{E}(X|\mathcal{D})$ is the \emph{almost surely}-unique random variable satisfying the two conditions:
\[
1) \ \mathbb{E}(X|\mathcal{D}) \text{ is measurable from } (\Omega,\mathcal{D}) \text{ to } (\mathbb{R},\mathcal{B}_{\mathbb{R}})
\]
\[
2) \int_D\mathbb{E}(X|\mathcal{D})d\mathbb{P}=\int_DXd\mathbb{P} \text{ for any D}\in\mathcal{D}
\]
Let $B\in \mathcal{F}$. The \emph{conditional probability} of B given $\mathcal{D}$ is defined to be:
\[
\mathbb{P}(B|\mathcal{D})= \mathbb{E}(\mathbb{I_B|\mathcal{D}})
\]
Suppose Y is measurable from $(\Omega,\mathcal{F},\mathbb{P})$ to  $(\Lambda,\mathcal{G})$. The \emph{conditional expectation of X given Y} is defined to be
\[
\mathbb{E}(X|Y)=E[X|\sigma(Y)]
\]
\subsection{Martingales}
Let $(\Omega,\mathcal{F},\mathbb{P})$ be a probability space.  A (discrete‐time) \emph{filtration} is an increasing sequence of sub‐$\sigma$‐algebras:
\[
  \mathcal{F}_0 \;\subset\;\mathcal{F}_1\;\subset\;\mathcal{F}_2\;\subset\;\cdots\;\subset\;\mathcal{F}.
\]
Suppose $\{X_n\}_{n\ge1}$ be a sequence of random variables and set
\[
  \mathcal F_n \;=\;\sigma\bigl(X_1,\dots,X_n\bigr)\,,\quad n\ge1, \ \mathcal F_0=\{\emptyset,\Omega\}
\]
We call a sequence of integrable random variables $\{M_n\}_{n\ge0}$ a \emph{martingale} with respect to the filtration $\{\mathcal F_n\}$ if:
\[
\text{1) }M_n \text{ is } \mathcal{F}_n\text{‐measurable for each n}  \ge 0
\]
\[
2)\ \mathbb{E}\bigl[\,|M_n|\,\bigr]<\infty \text{ for all n}\ge0,
\]
\[
3) \ \mathbb{E}\bigl[M_n\mid \mathcal F_{n-1}\bigr]=M_{n-1}\ \text{almost surely for each n}\ge1
\]
We can think of the $X_i$'s as telling us the $i$th outcome of some gambling process, and $M_n$ as the fortune of a gambler who places fair bets in varying amounts on the results of the coin tosses. 3) tells us that the expected value of the gambler's fortune at time n given all the information in the first $n-1$ flips of the coin is simply $M_{n-1}$, the actual value of the gambler's fortune before the $n$th round of the gambling process.
\subsubsection{Example (Partial Sum Process):}
If $X_n$ are independent random variables with $\mathbb{E}(X_n)=0$ for all $n \geq1$, then the \emph{partial sum process} given by taking $S_0=0$ and $S_n=X_1+\ldots+X_n$ for all $n \geq 1$ is a martingale with respect to $\{X_n\}_{n\geq1}$
\subsubsection{Example (Condition/Uncondition):}
If $X_n$ are independent random variables with $\mathbb{E}(X_n)=0, \ Var(X_n)=\sigma^2$ for all $n \geq1$, and setting $M_0=0$ and $M_n=S_n^2-n\sigma^2$ for all $n\geq1$ gives us a martingale with respect to $\{X_n\}_{n\geq1}$ \newline \newline
Measurability is trivial, so we focus on seperating the conditioned and unconditioned parts of the process:
\[
\mathbb{E}(M_n|X_1,\ldots,X_n)=\mathbb{E}(S^2_{n-1}+2S_{n-1}X_n+X_n^2-n\sigma^2|X_1,\ldots,X_n)
\]
$S^2_{n-1}$ is a function of $X_1,\ldots,X_{n-1}$, so its conditional expectation given $X_1,\ldots,X_{n-1}$ is just $S^2_{n-1}$. We note that $\mathbb{E}(X_n|X_1,\ldots,X_{n-1})=\mathbb{E}(X_n)=0$ which implies $\mathbb{E}(X^2_n|X_1,\ldots,X_{n-1})=\sigma^2$. \newline \newline
Hence,
\[
\mathbb{E}(S^2_{n-1}+2S_{n-1}X_n+X_n^2-n\sigma^2|X_1,\ldots,X_n)
\]
\[
=S^2_{n-1}+2S_{n-1}\mathbb{E}(X_n|X_1,\ldots,X_{n-1})+\sigma^2-n\sigma^2
\]
\[
=S^2_{n-1}+2S_{n-1}(0)+(n-1)\sigma^2
\]
\[
=S^2_{n-1}+(n-1)\sigma^2
\]


\section{References}
Athreya, K.B. and Lahiri, S.N. (2006) Measure Theory and Probability Theory. Springer, Berlin. \newline \newline
Durrett, R. (2019). Probability: Theory and Examples (5th ed.). Cambridge University Press. \newline \newline
Folland, G. B. (1999). Real Analysis: Modern Techniques and Their Applications (2nd ed.). Wiley. \newline \newline 

\end{document}
