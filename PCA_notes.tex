\documentclass{article}
\usepackage{graphicx} % Required for inserting images
\usepackage{amsmath,amssymb,amsthm}
\title{Principal Components Analysis}
\author{Alexander Yuan}
\date{September 13 2025}

\begin{document}
 
\maketitle

\section{Background}
\subsection{History}
\emph{Principal Components Analysis (PCA)} goes all the way back to Harold Hotelling in the 1933 when the cutting edge topics in statistics at the time were the multivariate methods we know today. It was first 
introduced as a technique for obtaining a smaller set of orthogonal linear projections of a single collection of correlated variables where the projections are ordered by decreasing variance. Note that it is also
called the \emph{Karhunen-Loève transform} in communications theory and \emph{empirical orthogonal functions} in atomspheric science.

\subsection{Eigendecomposition}
Let $\mathbf{A}$ be a square $n$ by $n$ matrix with $n$ linearly independent eigenvectors $e_i \text{ for } i = 1, \ldots , n$. Then $\mathbf{A}$ can be factored as 
$$
\mathbf{A = E \Lambda E^{-1}}
$$
where $\mathbf{E}$ is the square $n$ by $n$ matrix whose $i$'th column is the eigenvector $e_i$/ of $\mathbf{A}$ and $\mathbf{\Lambda}$ is the diagonal matrix whose diagonal elements are the corresponding eigenvalues 
i.e $\mathbf{\Lambda}_{i,i}= \lambda_i$. Note that this is a special case of the \emph{spectral theorem}.


\subsection{Singular Value Decomposition (SVD)}
Let $T \in \mathcal{L}_{V \to W}$ where $V$ and $W$ are two finite dimensional inner product spaces both over $\mathbb{C}$. Let $T^* \in \mathcal{L}_{W \to V}$ be the \emph{adjoint} of $T$ and recall that a map is called \emph{self-adjoint} 
if it equals its adjoint. From this, one can deduce that the eigenvalues of $T^*T$ are non-negative. \newline \newline 
Define the \emph{singular values} of $T$ to be the non-negative square roots of the eigenvalues of $T^*T$ in decreasing order, 
each included as many times as the dimension of the corresponding eigenspace of $T^*T$. Some important properties of $T^*T$ to be aware of is: 
$$
\text{i) }T^*T \text{ is self-adjoint and } \langle T^*Tv,v\rangle\geq 0 \text{ for all }v \in V
$$
$$
\text{ii) null } T^*T = \text{null } T
$$
$$
\text{iii) range }T^*T= \text{range }T^*
$$
$$
\text{iv) dim range }T = \text{dim range }T^* = \text{dim range }T^*T
$$
\newline \newline
We can also characterize positive singular values:
$$
\text{i) }T \text{ is injective } \iff \text{0 is not a singular value of }T
$$
$$
\text{ii) the number of positive singular values of $T$ = dim range }T  
$$
$$
\text{iii) }T \text{ is surjective} \iff \text{number of positive singular values of $T$ = dim }W
$$
\newline \newline
Before we state the result, note that SVD is a fundamental result as it shows every linear map from $V$ to $W$ has a clean description in terms of its singular values
and orthonormal lists in $V$ and $W$. Techniques exist for approximating eigenvalues and eigenvectors of positive operators which makes SVD very useful.
\newline \newline
Suppose $T \in \mathcal{L}_{V \to W}$ and let the positive singular values of $T$ be $s_1, s_2, \ldots, s_m$. Then there exists 
orthonormal lists $e_1, \ldots, e_m$ in $V$ and $f_1, \ldots f_m$ in $W$ such that 
$$
Tv = s_1\langle v,e_1\rangle f_1+ \ldots + s_m \langle v,e_m\rangle f_m
$$
for every $v \in V$.
\section{References}
Axler, S. (2024). Linear Algebra Done Right (4th ed.). Springer. \newline \newline
Hastie, T., Tibshirani, R., \& Friedman, J. (2009). The Elements of Statistical Learning: Data Mining, Inference, and Prediction (2nd ed.). Springer \newline \newline 
Ma, Y., \& Fu, Y. (Eds.). (2011). Manifold Learning Theory and Applications. CRC Press.
\end{document}